% Options for packages loaded elsewhere
\PassOptionsToPackage{unicode}{hyperref}
\PassOptionsToPackage{hyphens}{url}
%
\documentclass[
]{article}
\usepackage{lmodern}
\usepackage{amssymb,amsmath}
\usepackage{ifxetex,ifluatex}
\ifnum 0\ifxetex 1\fi\ifluatex 1\fi=0 % if pdftex
  \usepackage[T1]{fontenc}
  \usepackage[utf8]{inputenc}
  \usepackage{textcomp} % provide euro and other symbols
\else % if luatex or xetex
  \usepackage{unicode-math}
  \defaultfontfeatures{Scale=MatchLowercase}
  \defaultfontfeatures[\rmfamily]{Ligatures=TeX,Scale=1}
\fi
% Use upquote if available, for straight quotes in verbatim environments
\IfFileExists{upquote.sty}{\usepackage{upquote}}{}
\IfFileExists{microtype.sty}{% use microtype if available
  \usepackage[]{microtype}
  \UseMicrotypeSet[protrusion]{basicmath} % disable protrusion for tt fonts
}{}
\makeatletter
\@ifundefined{KOMAClassName}{% if non-KOMA class
  \IfFileExists{parskip.sty}{%
    \usepackage{parskip}
  }{% else
    \setlength{\parindent}{0pt}
    \setlength{\parskip}{6pt plus 2pt minus 1pt}}
}{% if KOMA class
  \KOMAoptions{parskip=half}}
\makeatother
\usepackage{xcolor}
\IfFileExists{xurl.sty}{\usepackage{xurl}}{} % add URL line breaks if available
\IfFileExists{bookmark.sty}{\usepackage{bookmark}}{\usepackage{hyperref}}
\hypersetup{
  pdftitle={MATH2070 Assignment},
  hidelinks,
  pdfcreator={LaTeX via pandoc}}
\urlstyle{same} % disable monospaced font for URLs
\usepackage[left=1cm,right=1cm,top=1cm,bottom=1cm]{geometry}
\usepackage{graphicx,grffile}
\makeatletter
\def\maxwidth{\ifdim\Gin@nat@width>\linewidth\linewidth\else\Gin@nat@width\fi}
\def\maxheight{\ifdim\Gin@nat@height>\textheight\textheight\else\Gin@nat@height\fi}
\makeatother
% Scale images if necessary, so that they will not overflow the page
% margins by default, and it is still possible to overwrite the defaults
% using explicit options in \includegraphics[width, height, ...]{}
\setkeys{Gin}{width=\maxwidth,height=\maxheight,keepaspectratio}
% Set default figure placement to htbp
\makeatletter
\def\fps@figure{htbp}
\makeatother
\setlength{\emergencystretch}{3em} % prevent overfull lines
\providecommand{\tightlist}{%
  \setlength{\itemsep}{0pt}\setlength{\parskip}{0pt}}
\setcounter{secnumdepth}{-\maxdimen} % remove section numbering
\usepackage{hyperref}
\usepackage{subfig}
\usepackage[utf8]{inputenc}
\def\tightlist{}

\title{MATH2070 Assignment}
\author{true}
\date{}

\begin{document}
\maketitle
\begin{abstract}
MATH2070 Assignment
\end{abstract}

\hypertarget{covariance-and-correlation}{%
\section{Covariance and Correlation}\label{covariance-and-correlation}}

\hypertarget{justify-the-use-of-the-log-return-rate.-what-are-the-advantages-of-using-it}{%
\subsection{Justify the use of the log return rate. What are the
advantages of using
it?}\label{justify-the-use-of-the-log-return-rate.-what-are-the-advantages-of-using-it}}

Based on the explanation and examples from
\href{https://investmentcache.com/magic-of-log-returns-concept-part-1/}{investmentcache}:

\begin{enumerate}
\def\labelenumi{\arabic{enumi}.}
\item
  Log returns can be added across time periods. In the case where our
  target price/index increases and then decreases by same amount, simple
  returns would present a \textbf{positive} total return while log
  returns would present a 0 return which is much more in accord with our
  cognizance.
\item
  Log returns follows normal distribution. In some certain areas, the
  stock price is assumed to follow a log normal distribution. Also,
  considering the fact that log requires that its corresponding value
  should never be less than 0, which is exactly one property of assets
  price, log return is even more preferable.
\end{enumerate}

\hypertarget{rebasing}{%
\subsection{Rebasing}\label{rebasing}}

As was shown in Figure\ref{one_ii}, the original indices have been
adjusted based on the value on Mar 1, 2007.

\begin{figure}[!htb]
\centering
\includegraphics{one_ii}
\caption{Rebased index values of each country from 2007 to 2020}\label{one_ii}
\end{figure}

\hypertarget{correlation-matrices}{%
\subsection{Correlation matrices}\label{correlation-matrices}}

\begin{figure}
\centering
\subfloat[GFC(03/01/2007 - 31/05/2010)]{%
\resizebox*{7cm}{!}{\includegraphics{GFC_cor}}}\hspace{5pt}
\subfloat[GFC-peak(02/09/2008 – 01/06/2009)]{%
\resizebox*{7cm}{!}{\includegraphics{GFC_peak}}}
\subfloat[INTERIM(01/06/2010 – 10/03/2020)]{%
\resizebox*{7cm}{!}{\includegraphics{INTERIM}}}\hspace{5pt}
\subfloat[COVID-19(11/03/2020 – 31/08/2020)]{%
\resizebox*{7cm}{!}{\includegraphics{COVID_19}}}
\subfloat[COVID-19-peak(11/03/2020 – 29/05/2020)]{%
\resizebox*{7cm}{!}{\includegraphics{COVID_19_peak}}}
\caption{Correlation matriices of each country for 5 periods} \label{corr_iii}
\end{figure}

Figure\ref{corr_iii} presents a \textbf{heatmap} based on the
correlation between each market over the five periods.

The \textbf{heatmap} basically indicates that, during the peak of
COVID---19, the 20 markets are relatively correlated with each other
with \emph{orange} and even \emph{red} dominating the plot. Conversely,
in the \textbf{INTERIM} period, most blue squares show that the
interaction among these markets is quite low.

The Global Financial Crisis (GFC) and even its peak time did not bring
much expected interaction to these markets. However, similar to what we
have described above, COVID-19 and its peak did tie them together,
probably more closely than ever before.

\hypertarget{histograms-of-the-correlation-coefficients}{%
\subsection{Histograms of the correlation
coefficients}\label{histograms-of-the-correlation-coefficients}}

Figure\ref{histo_iv} presents the histogram of \(\rho_{ij}\) of each
market. Note: we did include same \(\rho_{ij}\) twice. For example,
\(\rho_{12}\) and \(\rho_{21}\) are both plotted into each histogram.
But this does not affect the overall trend and relative quantity.

Similar to what we have found in Figure\ref{corr_iii}, here the right
skewed \(\rho_{ij}\) histograms is less significant during COVID-19 and
its peak period, which is not a pleasant plot at all because that means
the markets are more and more correlated with each other.

Also, we could not see significant difference as for the histogram of
GFC and its following peak.

\begin{figure}
\centering
\subfloat[GFC(03/01/2007 - 31/05/2010)]{%
\resizebox*{7cm}{!}{\includegraphics{histo1}}}\hspace{5pt}
\subfloat[GFC-peak(02/09/2008 – 01/06/2009)]{%
\resizebox*{7cm}{!}{\includegraphics{histo2}}}
\subfloat[INTERIM(01/06/2010 – 10/03/2020)]{%
\resizebox*{7cm}{!}{\includegraphics{histo3}}}\hspace{5pt}
\subfloat[COVID-19(11/03/2020 – 31/08/2020)]{%
\resizebox*{7cm}{!}{\includegraphics{histo4}}}
\subfloat[COVID-19-peak(11/03/2020 – 29/05/2020)]{%
\resizebox*{7cm}{!}{\includegraphics{histo5}}}
\caption{Correlation Coefficients $\rho_{ij}$ for $1 <i, j < 20$} \label{histo_iv}
\end{figure}

\hypertarget{portfolio-theory-2}{%
\section{Portfolio Theory 2}\label{portfolio-theory-2}}

\hypertarget{dollar-amount-invested-in-each-countrys-index-and-optimal-portfolio-p}{%
\subsection{Dollar amount invested in each country's index and optimal
portfolio
P*}\label{dollar-amount-invested-in-each-countrys-index-and-optimal-portfolio-p}}


\begin{table}[!htb]
\centering
{\begin{tabular}{lcccccc} 
  \hline
 & Country & Investment \\ 
  \hline
1 & Italy & -228000.00 \\ 
  2 & US & 1890000.00 \\ 
  3 & Australia & 126000.00 \\ 
  4 & UK & -972000.00 \\ 
  5 & Brazil & 26700.00 \\ 
  6 & Japan & 311000.00 \\ 
  7 & Russia & -260000.00 \\ 
  8 & China & 28800.00 \\ 
  9 & France & -569000.00 \\ 
  10 & Canada & -889000.00 \\ 
  11 & Korea & -492000.00 \\ 
  12 & Switzerland & 48500.00 \\ 
  13 & Spain & -815000.00 \\ 
  14 & India & 634000.00 \\ 
  15 & Mexico & -389000.00 \\ 
  16 & Germany & 958000.00 \\ 
  17 & Indonesia & 435000.00 \\ 
  18 & Netherlands & 1060000.00 \\ 
  19 & Saudi & -117000.00 \\ 
  20 & Turkey & 209000.00 \\ 
   \hline
\end{tabular}}
\caption{Investment in each market with 1,000,000 dollors.}
\label{money_invested}
\end{table}


Based on formulas from the textbook, we firstly calculated \textbf{a},
\textbf{b}, \textbf{c}, \textbf{d} and then with \emph{\(t=0.2\)}, the
optimal portfolio \(P^\star\) including the weight of each asset, its
\(\mu^{\star}\) and \(\sigma^{\star}\) are available.

Thus, we should invest as below with negative values meaning we should
short sell the corresponding market indices and positive values meaning
we should take a long position in terms of the corresponding market
indices. We get \(\mu^{\star} = 0.00138432\) and
\(\sigma^{\star} = 0.0166244\)

Partial codes are as below:

\begin{verbatim}
a = np.dot(ones_array_trans, returns_3_full_variance_annualized_inverse)
a = np.dot(a, ones_array)

b = np.dot(ones_array_trans, returns_3_full_variance_annualized_inverse)
b = np.dot(b, returns_3_full_mean_annualized)

c = np.dot(returns_3_full_mean_annualized_trans, 
returns_3_full_variance_annualized_inverse)
c = np.dot(c, returns_3_full_mean_annualized)

d=np.dot(a, c)
d = np.subtract(d, b**2)

alpha = (1/a) * returns_3_full_variance_annualized_inverse
alpha = np.dot(alpha, ones_array)

beta_real = np.dot(returns_3_full_variance_annualized_inverse,
np.subtract(returns_3_full_mean_annualized, (b/a)*ones_array))

x_weight  = np.add(alpha, 0.2*beta_real)
money_invested = 1000000 * x_weight
miu_star = (b+d*0.2)/a
varian = (1+d*0.2**2)/a
sigma_star = np.sqrt(varian)
\end{verbatim}

\hypertarget{plot}{%
\subsection{Plot}\label{plot}}

\begin{figure}[!htb]
\includegraphics{miusigma}
\caption{$\mu\sigma$-plane}\label{miusigma}
\end{figure}

Figure\ref{miusigma} plots things as below:

\begin{itemize}
\item
  All 20 country indices;
\item
  \textbf{MVF} (Minimum Variance Frontier) and \textbf{EF} (Efficient
  Frontier);
\item
  1000 random feasible portfolios;
\item
  Indifference curve of an investor with t = 0.2 and their optimal
  portfolio.
\end{itemize}

We should notice that some of the randomly generated portfolios are away
from EF even MVF. This \emph{inefficient} random portfolios derive from
the fact that our \textbf{``Randomly''} generated portfolios are not
\textbf{random} at all, at least not random enough. That may be caused
by the simplicity of the codes used to generate these portfolios:

\begin{verbatim}
while port_count < 1000:
    weights = np.random.random(num_assets)
    weights = weights/np.sum(weights)
    weights = weights.reshape((20, 1))
    # print(weights)
    varian_port_loop = np.dot(weights.transpose(), returns_3_full_variance_annualized)
    varian_port_loop = np.dot(varian_port_loop, weights)
    sigma_i = np.sqrt(varian_port_loop)
    if np.all([np.abs(weights) < 20*one_array_loop]) and (sigma_i < 0.1): 
        p_weights.append(weights) 
        p_vol.append(sigma_i)
        port_count = port_count + 1
\end{verbatim}

where, although we did have a constraint \(\mid{x_{nj}}\mid\leq20\) for
each of the i = 1, 2, \ldots, 20 country indices, the random weights
generated through above codes are rarely over 1 (test by uncomment codes
\texttt{print(weights)} above and comment the codes below). This means
that the short-selling is not even considered by the algorithm which is
partly not what happens in the real market and particularly not what
happens to the portfolios on EF. Therefore, some of these
\emph{randomly} portfolios are far away from the efficient frontier.

\hypertarget{portfolio-theory-3}{%
\section{Portfolio Theory 3}\label{portfolio-theory-3}}

\hypertarget{interim}{%
\subsection{INTERIM}\label{interim}}

\begin{table}[!htb]
\centering
{\begin{tabular}{lcccccc} 
  \hline
 & Country & Left & Right \\ 
  \hline
1 & Italy & -0.06 & Inf \\ 
  2 & US & -Inf & 0.00 \\ 
  3 & Australia & 2.24 & Inf \\ 
  4 & UK & 0.02 & Inf \\ 
  5 & Brazil & -Inf & 0.12 \\ 
  6 & Japan & -Inf & 0.01 \\ 
  7 & Russia & -0.05 & Inf \\ 
  8 & China & 0.66 & Inf \\ 
  9 & France & -0.19 & Inf \\ 
  10 & Canada & 0.04 & Inf \\ 
  11 & Korea & 0.03 & Inf \\ 
  12 & Switzerland & 0.27 & Inf \\ 
  13 & Spain & 0.02 & Inf \\ 
  14 & India & -Inf & -0.05 \\ 
  15 & Mexico & 0.06 & Inf \\ 
  16 & Germany & -Inf & 0.01 \\ 
  17 & Indonesia & -Inf & -0.04 \\ 
  18 & Netherlands & -Inf & -0.01 \\ 
  19 & Saudi & 0.1 & Inf \\ 
  20 & Turkey & -Inf & -0.05 \\ 
   \hline
\end{tabular}}
\caption{T range of short-selling for each country}
\label{trange_ss_table}
\end{table}

For this question, we need to consider:

\begin{itemize}
\item
  Which investors short sell?
\item
  Which indices they short sell?
\item
  Are there any country indices which no investors short sell or which
  all investors will short sell?
\end{itemize}

\begin{figure}[!htb]
\includegraphics{trange_ss}
\caption{T-range for short-selling}\label{trange_ss_png}
\end{figure}

Figure\ref{trange_ss_png} implies that investors with
\(0.27\leq t < 2.24\) will shortsell most of its assets at hand except
for the assets in \textbf{Australia}, \textbf{Germany}, \textbf{India},
\textbf{Indonesia}, \textbf{Japan}, \textbf{Netherlands},
\textbf{Turkey} and \textbf{US}. As \(>2.24\), investors will also take
a short position at assets in \textbf{Australia}. Also, investors with
\(t<-0.19\) would decide to take a short position at the eight assets
mentioned above while taking a long position at other assets. Based on
our research and plot, \textbf{if only investors have a sufficiently
high or low \(t\), they would short sell some certain indices.}

\hypertarget{covid-19-period}{%
\subsection{COVID-19 period}\label{covid-19-period}}


\begin{table}[!htb]
\centering
{\begin{tabular}{lcccccc} 
  \hline
 & Country & Left & Right \\ 
  \hline
1 & Italy & -Inf & 0.00 \\ 
  2 & US & -Inf & 0.00 \\ 
  3 & Australia & 0.01 & Inf \\ 
  4 & UK & 0.00 & Inf \\ 
  5 & Brazil & -Inf & 0.26 \\ 
  6 & Japan & -Inf & -0.18 \\ 
  7 & Russia & -Inf & 0.06 \\ 
  8 & China & -Inf & -0.07 \\ 
  9 & France & -0.01 & Inf \\ 
  10 & Canada & 0.00 & Inf \\ 
  11 & Korea & -Inf & 0.04 \\ 
  12 & Switzerland & -Inf & -0.05 \\ 
  13 & Spain & 0.00 & Inf \\ 
  14 & India & -Inf & 0.02 \\ 
  15 & Mexico & 0.04 & Inf \\ 
  16 & Germany & -Inf & 0.01 \\ 
  17 & Indonesia & 0.03 & Inf \\ 
  18 & Netherlands & 0.03 & Inf \\ 
  19 & Saudi & -Inf & -0.06 \\ 
  20 & Turkey & 0.09 & Inf \\ 
   \hline
\end{tabular}
\caption{T range of short-selling for each country (Covid-19 period) }
}
\label{trange_co_table}
\end{table}


The detailed values are available in Table\ref{trange_co_table} and
similarly based on this table, we get the following plot.

\begin{figure}[!htb]
\includegraphics{trange_co}
\caption{T-range for short-selling (Covid-19 period)}\label{trange_co_png}
\end{figure}

As was shown in Figure\ref{trange_co_png}, as \(0.09 \leq t \leq 0.26\),
investors short sell most of its indices except for it in
\textbf{China}, \textbf{Italy}, \textbf{Japan}, \textbf{Korea},
\textbf{Russia} , \textbf{Saudi}, \textbf{Switzerland} and \textbf{US}.
Conversely, investors with \(-0.18< t\leq -0.07\) would decide to take a
long position at assets mentioned above.

Compared to the first plot (See Figure\ref{trange_ss_png}), there are
some significant distinctions. For example, investors with
\(t\geq 2.24\) would shortsell Australian index but during Covid-19, t =
0.09 would be enough to urge them to take a short position. Similar
situation happens to \textbf{China} and \textbf{France}.

Also, some investors actually took reversed position during Covid-19.
For example, previously, investors with \(t>0.66\) would take a short
position at Chinese market index whereas they would only do the same
thing with \(t<-0.07\) during Covid-19. Similar things also happen to
\textbf{Indonesia}, \textbf{Italy}, \textbf{Korea},
\textbf{Netherlands}, \textbf{Russia}, \textbf{Saudi},
\textbf{Switzerland} and \textbf{Turkey}. Notably,
Figure\ref{trange_ss_png} and Figure\ref{trange_co_png} indicate that
even during Covid-19, investors did not changes its attitude to US
market index significantly with both t around less than 0.

\hypertarget{adding-a-riskless-cash-fund-and-constructing-the-market-portfolio}{%
\section{Adding a Riskless Cash Fund and Constructing the Market
Portfolio}\label{adding-a-riskless-cash-fund-and-constructing-the-market-portfolio}}

\hypertarget{new-allocation-of-their-investment}{%
\subsection{New allocation of their
investment}\label{new-allocation-of-their-investment}}

\label{tp_w_weight_tilde} We got the daily return of this Riskless Cash
Fund using \texttt{r0/250}. Then based on formulas from the textbook, we
calculate the weight of each risky asset and the cash fund in the
portfolio by

\begin{verbatim}
returns_3_full_mean_annualized_tilde = np.subtract(returns_3_full_mean_annualized,
r0*ones_array)
x_weight_tilde = 0.2 * np.dot(returns_3_full_variance_annualized_inverse,
returns_3_full_mean_annualized_tilde)
x_weight_0 = 1- np.dot(np.transpose(x_weight_tilde), ones_array)
\end{verbatim}

where we obtain the following Table\ref{x_weight_tilde_plus0}. The
negative weight means that investors with \(t=0.2\) should short sell
the corresponding asset while the other way around for positive weight.
This riskless cash fund would approximately account for 5\% of our
portfolio.

\begin{table}[!htb]
\centering
{\begin{tabular}{lcccccc} 
  \hline
 & Country & Weight \\ 
  \hline
1 & Italy & -0.22 \\ 
  2 & US & 1.89 \\ 
  3 & Australia & 0.12 \\ 
  4 & UK & -0.98 \\ 
  5 & Brazil & 0.03 \\ 
  6 & Japan & 0.31 \\ 
  7 & Russia & -0.26 \\ 
  8 & China & 0.03 \\ 
  9 & France & -0.55 \\ 
  10 & Canada & -0.90 \\ 
  11 & Korea & -0.50 \\ 
  12 & Switzerland & 0.04 \\ 
  13 & Spain & -0.82 \\ 
  14 & India & 0.63 \\ 
  15 & Mexico & -0.40 \\ 
  16 & Germany & 0.96 \\ 
  17 & Indonesia & 0.43 \\ 
  18 & Netherlands & 1.06 \\ 
  19 & Saudi & -0.12 \\ 
  20 & Turkey & 0.21 \\ 
  21 & Cash Fund & 0.05 \\ 
   \hline
\end{tabular}
\caption{Weight of each asset including cash fund}}
\label{x_weight_tilde_plus0}
\end{table}

\hypertarget{capital-market-line-and-the-tangency-portfolio}{%
\subsection{Capital Market Line and the tangency
portfolio}\label{capital-market-line-and-the-tangency-portfolio}}

\label{CML and TP}

\begin{figure}[!htb]
\begin{center}
\includegraphics[width=1\textwidth]{textbook_CML}
\end{center}
\caption{CML and MVF}\label{textbook_CML}
\end{figure}

According to Buchen and Ivers (2020), CML (Capital Market Line is),
which is denoted by
\(\hat\sigma = \frac{\hat\mu - r_0}{\sigma_0\sqrt{d}}\), is the
``degenerate form of the efficient frontier''. It is the line \(P_0M\)
as is shown in Figure\ref{textbook_CML} from the textbook.

The fact that CML is tangential to EF has been proved in the textbook.
Based on the summary of Buchen and Ivers, we have the following
observations:

\begin{itemize}
\item
  \(r_{riskless-asset} < \frac{b}{a}\) is the most practical case where
  CML has a positive slope and M is the \emph{Tangency Portfolio}. An
  investor would not take any risky assets into consideration except for
  riskless asset at \(\hat\sigma = 0\) which is understandable because
  they did not have risky assets in their portfolio while risk less
  asset offers 0 \(\sigma\). Similarly, riskless assets will be dropped
  totally in our \emph{Tangency Portfolio M}.
\item
  \(r_{riskless-asset} > \frac{b}{a}\) represents a CML with negative
  slope and the \emph{Tangency Portfolio M} is at the lower part of MVF,
  a symmetrical point of M at the upper part.
\item
  In the remaining case, Buchen and Ivers (2020) found that CML is the
  asymptote to MVF.
\end{itemize}

In our case, where \(r_0 = 0.00001\) and \(\frac{b}{a} = 0.00014524\),
we got a practical case at hand. Our \emph{Tangency Portfolio M} has
\(\sigma_M = 0.0174574\) and \(\mu_M = 0.00145337\) which would be
plotted in the next part of this assignment.

\hypertarget{gdp}{%
\subsection{GDP}\label{gdp}}

We retrieve \textbf{Current Price Gross Domestic Product} of each
country with seasonally adjusted from
\href{https://fred.stlouisfed.org/}{Federal Reserve Bank of St.~Louis}.
We decided to use \textbf{Index(Scale value to the chosen date)} as the
original data and set \textbf{2009-06-01} as the end of US recession
date and the value of that day is the base value. All of these is done
on \href{https://fred.stlouisfed.org/}{Federal Reserve Bank of
St.~Louis}.

Note that we did not get the quarterly indicator of \textbf{Saudi
Arabia}. After getting its annual data, we calculate its log returns and
using the equation below

\begin{equation}
\ (1+r_{quarterly})^4 = (1 + r_{yearly})
\end{equation}

to get the quarterly return.

Since we have collected the original data, similar steps had been
executed in our python code. Partial codes are as below:

\begin{verbatim}
one_array_20 = np.ones((20, 1), dtype=np.int32)
one_array_20_tran = np.transpose(one_array_20)
a_gdp = np.dot(one_array_20_tran, returns_quaterly_full_variance_inv)
a_gdp = np.dot(a_gdp, one_array_20)

b_gdp = np.dot(one_array_20_tran, returns_quaterly_full_variance_inv)
b_gdp = np.dot(b_gdp, returns_quaterly_full_mean)

c_gdp = np.dot(returns_quaterly_full_mean_tran, returns_quaterly_full_variance_inv)
c_gdp = np.dot(c_gdp, returns_quaterly_full_mean)

d_gdp = np.dot(a_gdp, c_gdp)
d_gdp = np.subtract(d_gdp, b_gdp**2)

alpha_gdp = (1/a_gdp) * returns_quaterly_full_variance_inv
alpha_gdp = np.dot(alpha_gdp, one_array_20)

beta_gdp = np.dot(returns_quaterly_full_variance_inv, np.subtract(returns_quaterly_full_mean, 
(b_gdp/a_gdp)*one_array_20))

x_weight_gdp =np.add(alpha_gdp, beta_gdp * 0.2)

miu_gdp = (c_gdp-b_gdp*r0)/(b_gdp-a_gdp*r0)
\end{verbatim}

We are required to derive a market portfolio from this new dataset.
Within the following few steps, we use the formulas from textbook to get
the \emph{Tangency Portfolio} of this quarterly gdp dataset. The reasons
are as follows:

\begin{itemize}
\item
  Buchen and Ivers (2020) suggest it is irresistible to equate the
  \emph{Tangency Portfolio} with the \emph{Market Portfolio} which, as
  they stated, is the key assumption of \emph{Capital Asset Pricing
  Model}. Therefore, firstly we got the theoretical support for this
  equivalent relationship.
\item
  Also, as what we have discussed in \ref{CML and TP}, at Point M,
  \emph{Tangency Portfolio} actually holds 0 \textbf{riskeless asset} so
  that we could confidently believe this point can kind of represent our
  target \textbf{Market Portfolio}.
\end{itemize}

\hypertarget{capital-market-line}{%
\section{Capital Market Line}\label{capital-market-line}}

\begin{figure}[!htb]
\includegraphics[width=1\textwidth]{CML_other}
\caption{CML+Tangency Portfolio+Market Portfolio+$P_0$+EF}\label{CML_other}
\end{figure}

The final plot is shown in Figure\ref{CML_other}.

Basically, CML was plotted using Equation\ref{miuhat} and
Equation\ref{CML_formula} with \(t \in [-0.4,1]\) \begin{equation}
\hat\mu = r_0 + d\sigma_0^2t
\label{miuhat}
\end{equation}

\begin{equation}
\hat\sigma = \frac{\hat\mu - r_0}{\sigma_0\sqrt{d}}
\label{CML_formula}
\end{equation}

where \(\sigma_0\) is the volatility of the portfolio with \(r_0\)
return on EF for \textbf{risky assets only} and a, b, c and d are
exactly what we have calculated before in order to get
Figure\ref{miusigma}.

The \textbf{Market Portfolio} comes from the previous calculation. Note
that we have change it into daily basis by using codes
\texttt{sigma\_gdp/(250/4)} and \texttt{miu\_gdp/(250/4)}

If we have 100 million dollars available, considering there exists the
riskless cash fund, the new weight of each asset is similar to the
\textbf{x\_weight\_tilde}, which we have calculated here (Refer to
\ref{tp_w_weight_tilde}).

Therefore, we have the following table:

\begin{table}[!htb]
\centering
{\begin{tabular}{lcccccc} 
  \hline
 & Country & Investment \\ 
  \hline
1 & Italy & -22.5 \\ 
  2 & US & 189.0 \\ 
  3 & Australia & 11.9 \\ 
  4 & UK & -97.9 \\ 
  5 & Brazil & 2.9 \\ 
  6 & Japan & 31.3 \\ 
  7 & Russia & -25.7 \\ 
  8 & China & 2.7 \\ 
  9 & France & -55.4 \\ 
  10 & Canada & -90.2 \\ 
  11 & Korea & -49.6 \\ 
  12 & Switzerland & 3.8 \\ 
  13 & Spain & -81.9 \\ 
  14 & India & 62.7 \\ 
  15 & Mexico & -39.7 \\ 
  16 & Germany & 96.0 \\ 
  17 & Indonesia & 43.1 \\ 
  18 & Netherlands & 106.0 \\ 
  19 & Saudi & -12.3 \\ 
  20 & Turkey & 20.6 \\ 
   \hline
\end{tabular}
\caption{Investment in each index. \textsuperscript{a}Negative values mean that we short sell the corresponding assets.}}
\label{investment_100.csv}
\end{table}

\hypertarget{security-market-line}{%
\section{Security Market Line}\label{security-market-line}}

\begin{figure}[!htb]
\begin{center}
\includegraphics[width=0.7\textwidth]{SML_manually}
\end{center}
\caption{Security Market Line}\label{SML}
\end{figure}

As was shown in Figure\ref{SML}, all assets are on Security Market Line.
And Figure\ref{betagreater} indicates that only \(\beta_{US} > 1\) while
\(\beta_{Japan}\) and \(\beta_{India}\) are very close to 1. Notably,
Russia and Spain both presented a negative \(\beta\).

Buchen and Ivers suggested in 2020 that \emph{Portfolio Theory} is
``about managing risk''. \(\beta\) is one measure of the risk of assets
relative to the market portfolio (Buchen \& Ivers, 2020).

\begin{figure}[!htb]
\begin{center}
\includegraphics[width=0.6\textwidth]{betagreater}
\end{center}
\caption{$\beta$ of 20 countries}\label{betagreater}
\end{figure}

A higher \(\beta_{US}\) means a higher return as was stated by
Table\ref{beta_countries_name} whereas, concurrently, an investor has to
undergo the extra volatility.

However, a relatively low \(\beta\) would probably offer a negative
return (such as \textbf{Italy}, \textbf{Russia} and \textbf{Spain}).

Then, based on the \emph{Portfolio Theory}, risk management does not
suggest we should pursue 0 risk. On the contrary, an appropriate
portfolio taking both return and risk into consideration would be
preferable. For example, \textbf{Saudi} with the lowest \(\beta\) but
offers almost the same return as UK might benefit our portfolio.
Undoubtedly, there are many other reasons in the real world affecting
the return of our portfolio so that an investor with \emph{Portfolio
Theory} and a global vision might outperform its counterparts.

\begin{table}[!htb]
\caption{$\beta$, $r_{daily}$ and $r_{yearly}$}
{\begin{tabular}{lcccccc} 
    \hline
 & Country & Beta & Yearly Return & Daily Return \\ 
  \hline
1 & Italy & -0.16 & -0.01 & -0.00 \\ 
  2 & US & 1.31 & 0.10 & 0.00 \\ 
  3 & Australia & 0.37 & 0.03 & 0.00 \\ 
  4 & UK & 0.16 & 0.01 & 0.00 \\ 
  5 & Brazil & 0.49 & 0.04 & 0.00 \\ 
  6 & Japan & 0.95 & 0.07 & 0.00 \\ 
  7 & Russia & -0.36 & -0.02 & -0.00 \\ 
  8 & China & 0.50 & 0.04 & 0.00 \\ 
  9 & France & 0.35 & 0.03 & 0.00 \\ 
  10 & Canada & 0.32 & 0.02 & 0.00 \\ 
  11 & Korea & 0.21 & 0.02 & 0.00 \\ 
  12 & Switzerland & 0.49 & 0.04 & 0.00 \\ 
  13 & Spain & -0.35 & -0.02 & -0.00 \\ 
  14 & India & 0.99 & 0.07 & 0.00 \\ 
  15 & Mexico & 0.33 & 0.03 & 0.00 \\ 
  16 & Germany & 0.74 & 0.06 & 0.00 \\ 
  17 & Indonesia & 0.83 & 0.06 & 0.00 \\ 
  18 & Netherlands & 0.54 & 0.04 & 0.00 \\ 
  19 & Saudi & 0.10 & 0.01 & 0.00 \\ 
  20 & Turkey & 0.82 & 0.06 & 0.00 \\ 
   \hline
\end{tabular}}\label{beta_countries_name}
\end{table}

\hypertarget{references}{%
\section{References}\label{references}}

Buchen, P. W., \& Ivers, D. J. (2020). \emph{MATH2070 and MATH2970
Optimization and Financial Mathematics Lecture Notes.} \hangindent=2em
Sydney: The University of Sydney School of Mathematics and Statistics.

E.G. (2018, October 4). \emph{Magic of Log Returns: Concept -- Part 1.}
\hangindent=2em Retrieved from Investment Cache:
\url{https://investmentcache.com/magic-of-log-returns-concept-part-1/}

\end{document}
